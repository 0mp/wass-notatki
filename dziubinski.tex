\section{Dziubiński}

\subsection{Bonacich}


\begin{enumerate}
  \item
    \wassterm{benefit function}
    $
    b(0) = 0,
    b' > 0,
    b'' < 0
    $

  % \item
  %   intuition for
  %   \wassterm{positive externalities}
  %   and
  %   \wassterm{negative externalities}:

\end{enumerate}

\subsection{Stable Equilibria}

\begin{enumerate}
 \item Jeżeli $|\lambda_{min}(\textbf{G})| < 1/\delta$, istnieje \wassterm{unikalne Nash equilibrium} i jest ono \wassterm{stabilne}.
 \item Equilibrium z aktywnymi agentami A (i pozostali agenci są silnie nieaktywni\footnote{cokolwiek to znaczy - chyba ich effort musi być $< 0$}) jest \wassterm{stabilne} wtw, gdy  $|\lambda_{min}(\textbf{G}_{A})| < 1/\delta$.
 \item Jeżeli $|\lambda_{min}(\textbf{G})| > 1/\delta$, to może istnieć wiele equilibria i w skład wszystkich equilibria wchodzą agenci nieaktywni.
 \item We say a \wassterm{Nash equilibrium} \textbf{x} is \wassterm{asymptotically stable} when this system converges back to \textbf{x} following any small enough perturbation.
\end{enumerate}

\subsection{Gry komplementarne i substytutywne}
\begin{enumerate}
 \item
 	\wassterm{Komplementarność}: Dla każdego $i$, $a_i \geq a'_i$, $a_{-i} \geq a'_{-i} $:
 	$$u_i(a_i, a_{-i}) - u_i(a'_i, a_{-i}) \geq  u_i(a_i, a'_{-i}) - u_i(a'_i, a'_{-i}).$$
 	\wassterm{Substytutywność}: Dla każdego $i$, $a_i \geq a'_i$, $a_{-i} \geq a'_{-i} $:
 	$$u_i(a_i, a_{-i}) - u_i(a'_i, a_{-i}) \leq  u_i(a_i, a'_{-i}) - u_i(a'_i, a'_{-i}).$$
 \item Consider a game of strategic complements such that:
  \begin{itemize}
   \item for every player i, and specification of strategies of the other players, $a_{-i} \in A_{-i}$, player $i$ has a
nonempty set of best responses $BR_i(a_{-i})$ that is a closed sublattice of the complete lattice $A_i$, and
   \item for every player $i$, if $a'_{-i} \geq a_{-i}$, then $sup_{i} BR_{i}(a'_{-i}) \geq sup_{i} BR_{i}(a-{-i})$ and $inf_i BR_i(a'_{-i}) \geq inf_i BR_i(a_{-i})$.
  \end{itemize}
  An equilibrium exists and the set of equilibria form a (nonempty) complete lattice.
\end{enumerate}

\subsection{Inne}
\begin{enumerate}
 \item  A differentiable function $f$ is (strictly) concave on an interval if and only if its derivative function $f'$ is (strictly) monotonically decreasing on that interval, that is, a concave function has a non-increasing (decreasing) slope.
 \item If $f$ is twice-differentiable, then $f$ is concave if and only if $f''$ is non-positive (or, if the acceleration is non-positive). If its second derivative is negative then it is strictly concave, but the opposite is not true.
\end{enumerate}

\subsection{Zadanie 1}
\begin{enumerate}
    \item
      % (chyba wystarczy że nie pogarsza/polepsza)
      \wassterm{Positive externalities} są, gdy podbicie wartości przez innego gracza polepsza lub nie zmienia naszego wyniku:
      $s_i \geq s_i' \rightarrow u_j(s_i, s_{-i}') \geq u_j(s'_i, s_{-i})$
    \item
      \wassterm{Negative externalities} są, gdy podbicie wartości przez innego gracza pogarsza lub nie zmienia naszego wyniku:
      $s_i \geq s_i' \rightarrow u_j(s_i, s_{-i}') \leq u_j(s'_i, s_{-i})$
    \item
        \wassterm{concave} functions and \wassterm{convex} functions
        \begin{enumerate}
            \item
                $e_i \geq e'_i, e_{-i} \geq e'_i$,
            \item
                $ u(e_i, e_{-i}) u(e'_i, e_{-i}) ?  u(e_i, e'_{-i}) u(e'_i, e'_{-i}) $
            \item
                $ f(A) - f(B) ?  f(A - \varepsilon_C) - f(B - \varepsilon_D) $
            \item
                \wassterm{strategiczna substytutywność}---\wassterm{concave} function
            \item
                \wassterm{strategiczna komplementarność}---\wassterm{convex} function
        \end{enumerate}
      \item
        Wybranie $1$-ki jest zawsze bez sensu, bo gdy suma sąsiadów minus koszt
        $>= 0$, to się opłaca $2$ niemniej niż jeden, a gdy mniejsza, to się
        opłaca bardziej $0$, niż jeden.

        Najlepszą, w sensie wysiłków, \wassterm{równowagę Nasha} liczymy
        algorytmem \wassterm{best response dynamics} startując od stanu, gdzie
        wszyscy wybrali $2$.

        Gdybyśmy chcieli najgorszą \wassterm{równowagę Nasha}, to startujemy od
        stanu gdzie są same zera. Użycie algorytmu z samymi zerami dla szukania
        maksymalnej \wassterm{równowagi Nasha} jest błędem, bo znajdzie
        równowagę z samymi zerami, a w maksymalnej \wassterm{równowadze Nasha}
        jest zwykle kilka dwójek. Do tej równowagi nie da się dotrzeć ze stanu
        z samymi zerami, bo dwójki opłacają się, gdy inni już wybrali dwójki.

      \item
        Czy zawsze istnieje \wassterm{Nash equilibrium}? Tak, na mocy \wassterm{twierdzenia o równowagach w grach supermodularnych}.
\end{enumerate}

\subsection{Zadanie 2}
\begin{enumerate}
  \item
    jeżeli $\alpha_{min}(G)$ jest większe niż $-1/\delta$, to jest dokładnie jedno
    \wassterm{Nash equilibrium}
    %TODO: \item Wzór na liczenie \wassterm{eigenvalue(G)}: $ $
  \item Strategia \textbf{e} jest \wassterm{Nash equilibrium} wtw, gdy
    \begin{enumerate}
      \item $ (\textbf{I} + \delta \textbf{G}_{A})\overline{e}_{A} = 1$
      \item $ \delta\textbf{G}_{N-A,A}\overline{e}_A \geq 1.$
    \end{enumerate}

    gdzie $\overline{e}_A$ to profil strategii, potencjalnie \wassterm{Nash equilibrium}.

  \item
    Wzór na \wassterm{centralność Bonacicha}:
    $C(G, \alpha, \beta) = (I - \alpha G)^{-1} \beta G \cdot \overline{1}$.

  \item
    Centralność Bonacicha.

    Jeżeli $\overline{e} = \overline{1} - \delta C(G_A, -\delta, 1) \geq \overline{0}$,
    to $\overline{e}$  jest \wassterm{Nash equilibrium}.

    Wzór ten można stosować dla każdego węzła oddzielnie.

  \item
    $G$ ma \wassterm{Nash equilibrium} (\textit{znajdowalne w PTIME}), gdy
    $(I+\delta G)$ jest dodatnio określone.

    Test na dodatnią określoność: $\lambda_{min}(G) > \frac{-1}{\delta}$.
\end{enumerate}

\subsection{Zadanie 3}
\begin{enumerate}
  \item
    \wassterm{pairwise stable} (\wassterm{bilateralna stabilność}) $\leftrightarrow$
    \begin{enumerate}
      \item
        każde usunięcie nie poprawi wyniku żadnej ze stron (nikomu nie opłaca się kasować krawędzi),
      \item
        jeżeli dodanie poprawi coś jednemu, to drugi straci
         (żadna para nie chce robić krawędzi).
    \end{enumerate}

     Intuicja jest taka, że do usunięcia nie potrzeba zgody obu stron. Jeżeli
     jednej się nie podoba, to usuwa krawędź, więc nikomu nie może się opłacać,
     aby warunek był spełniony. Aby dodać, to ten co zyska musi przekonać
     drugiego, że nie straci, bo potrzebuje jego zgody.

     Więcej własności:
    \begin{enumerate}
      \item
        Jedyne \wassterm{BS} sieci spójne to drzewa.
    \end{enumerate}

   \item
     \wassterm{strongly stable}:
     \begin{enumerate}
       \item
         Dla wszystkich podzbiorów wierzchołków w grafie, dodawanie krawędzi
         wewnątrz grupy i usuwanie krawędzi wychodących z grupy nie polepsza
         sytuacji.
       \item
         Inaczej: żadnemu podzbiorowi wierzchołków nie opłaca się przerabiać sieci.
     \end{enumerate}
\end{enumerate}

% vim: filetype=tex softtabstop=2 shiftwidth=2 tabstop=2 expandtab
