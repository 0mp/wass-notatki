\documentclass{article}
\usepackage{hyperref}
\usepackage{amsmath}
\usepackage{amssymb}
\usepackage{xcolor}
\usepackage[utf8]{inputenc}
\usepackage{polski}

\newcommand\wassterm[1]{{\color{blue}{#1}}}

\begin{document}

\title{Notatki ze wstępu do analizy sieci społecznych}
\author{BTW, check out \href{https://freebsd.org}{FreeBSD} and \href{https://joinmastodon.org}{Mastodon}}

\maketitle

\begin{abstract}
\end{abstract}

\section{Apt}

\subsection{Równowagi Nasha i optima społeczne}

\begin{enumerate}

  \item a 
    \wassterm{strictly dominant strategy}
    is a unique 
    \wassterm{Nash equilibrium}

  \item
    a 
    \wassterm{dominant strategy}
    is a 
    \wassterm{Nash equilibrium}

  \item
    $s$ is 
    \wassterm{Pareto efficient} 
    if for no $s'$: 
    $
    \forall_i p_i(s') \geq p_i(s) 
    \wedge
    \exists_i p_i(s') > p_i(s) 
    $

  \item
    \wassterm{social welfare} 
    of $s$: $\sum^n_{j=1}p_j(s)$

  \item
    \wassterm{social optimum} 
    is a maximal
    \wassterm{social welfare} 

    \begin{enumerate}
      \item
        each 
        \wassterm{social optimum} 
        is 
        \wassterm{Pareto efficient}
    \end{enumerate}

  \item
    \wassterm{public goods game}

  \item
    \wassterm{tragedy of the commons}

\end{enumerate}

\subsection{Gry z potencjałem i gry zagęszczeniowe}

\begin{enumerate}

  \item
    \wassterm{best response dynamics}
    is an algorithm to find
    \wassterm{Nash equilibrium}

    \begin{enumerate}
      \item
        if every player has a 
        \wassterm{strictly dominant strategy}
        then all
        \wassterm{best response dynamics}
        terminate after at most $n$ steps and their outcomes are unique
      \item
        \wassterm{best response dynamics}
        may miss a
        \wassterm{Nash equilibrium}
    \end{enumerate}
  \item
    $P: S_1\times \cdot \times S_n \rightarrow \mathbb{R}$ is an
    \wassterm{exact potential function}
    for $G$ if
    $$
    \forall_i \forall_{s_{-i} \in S_{-i}} \forall_{s_i, s_i' \in S_i}
    p_i(s_i, s_{-i}) -
    p_i(s_i', s_{-i}) = 
    P(s_i, s_{-i}) -
    P(s_i', s_{-i})
    $$
    \begin{enumerate}
      \item
        \wassterm{potential game}
        has an
        \wassterm{exact potential function}
      \item
        \wassterm{exact potential function}
        count the number of defecting players

      \item
        all
        \wassterm{best response dynamics}
        terminate for a finite
        \wassterm{potential game}

      \item
        every
        \wassterm{congestion game}
        is a
        \wassterm{potential game}
        so it has a
        \wassterm{Nash equilibrium}

        \begin{enumerate}
          \item
            \wassterm{price of stability}
            $$
            \frac{\text{
              \wassterm{social cost}
            of the best
            \wassterm{Nash equilibrium}
            }}{{\text{
              \wassterm{social cost}
            of the
            \wassterm{social optimum} 
            }}}
            $$
        \end{enumerate}
    \end{enumerate}

\end{enumerate}

\subsubsection{weakly acyclic games}

\begin{enumerate}
  \item
    \wassterm{improvement path}
    is a maximal sequence of 
    \wassterm{profitable deviation}s
  \item
    $G$ has the
    \wassterm{finite improvement property}
    (FIP) if for any
    \wassterm{joint strategy}
    every 
    \wassterm{improvement path}
    that starts at it is finite

  \item
    a game is
    \wassterm{weakly acyclic}
    if for any 
    \wassterm{joint strategy}
    there exists a finite
    \wassterm{improvement path}
    that starts at it

  \item
    every game on a simple cycle is
    \wassterm{weakly acyclic}

    % TODO: Dokończ czytać "notatki z wykładu".


\end{enumerate}

\subsection{Diffusion in Social Networks with Competing Products}

\subsection{Social Networks Games}

\begin{enumerate}
  \item
    \wassterm{Nash equilibrium}
    exists and can be computed in polynomial time if there are at most two products
  \item
    in a DAG, a non-trivial 
    \wassterm{Nash equilibrium}
    always exists
\end{enumerate}



% End of Apt

\section{Dziubiński}

\begin{enumerate}
  \item
    jeżeli $\alpha_{min}(G)$ jest większe niż $-1/\delta$, to jest dokładnie jedno
    \wassterm{Nash equilibrium}
  %TODO: \item Wzór na liczenie \wassterm{eigenvalue(G)}: $ $
  \item Strategia \textbf{e} jest \wassterm{Nash equilibrium} wtw, gdy 
  	\begin{itemize}
	  \item $ (\textbf{I} + \delta \textbf{G}_{A})\textbf{x}_{A} = 1$
  	  \item $ \delta\textbf{G}_{N-A,A}\textbf{x}_A \geq 1.$
  	\end{itemize}
\end{enumerate}

\subsection{Bonacich}


\begin{enumerate}
  \item
    \wassterm{benefit function}
    $
    b(0) = 0, 
    b' > 0,
    b'' < 0
    $

  \item
    intuition for 
    \wassterm{positive externalities}
    and
    \wassterm{negative eternalities}:

\end{enumerate}

\subsection{Stable Equilibria}

\begin{enumerate}
 \item Jeżeli $|\lambda_{min}(\textbf{G})| < 1/\delta$, istnieje \wassterm{unikalne Nash equilibrium} i jest ono \wassterm{stabilne}.
 \item Equilibrium z aktywnymi agentami A (i pozostali agenci są silnie nieaktywni\footnote{cokolwiek to znaczy - chyba ich effort musi być $< 0$}) jest \wassterm{stabilne} wtw, gdy  $|\lambda_{min}(\textbf{G}_{A})| < 1/\delta$.
 \item Jeżeli $|\lambda_{min}(\textbf{G})| > 1/\delta$, to może istnieć wiele equilibira i w skład wszystkich equilibria wchodzą agenci nieaktywni.
 \item We say a \wassterm{Nash equilibrium} \textbf{x} is \wassterm{asymptotically stable} when this system converges back to \textbf{x} following any small enough perturbation.
\end{enumerate}

\subsection{Gry komplementarne i substytutywne}
\begin{enumerate}
 \item 
 	\wassterm{Komplementarność}: Dla każdego $i$, $a_i \geq a'_i$, $a_{-i} \geq a'_{-i} $:
 	$$u_i(a_i, a_{-i}) - u_i(a'_i, a_{-i}) \geq  u_i(a_i, a'_{-i}) - u_i(a'_i, a'_{-i}).$$
 	\wassterm{Substytutywność}: Dla każdego $i$, $a_i \geq a'_i$, $a_{-i} \geq a'_{-i} $:
 	$$u_i(a_i, a_{-i}) - u_i(a'_i, a_{-i}) \leq  u_i(a_i, a'_{-i}) - u_i(a'_i, a'_{-i}).$$
 \item Consider a game of strategic complements such that:
  \begin{itemize}
   \item for every player i, and specification of strategies of the other players, $a_{-i} \in A_{-i}$, player $i$ has a
nonempty set of best responses $BR_i(a_{-i})$ that is a closed sublattice of the complete lattice $A_i$, and
   \item for every player $i$, if $a'_{-i} \geq a_{-i}$, then $sup_{i} BR_{i}(a'_{-i}) \geq sup_{i} BR_{i}(a-{-i})$ and $inf_i BR_i(a'_{-i}) \geq inf_i BR_i(a_{-i})$.
  \end{itemize}
  An equilibrium exists and the set of equilibria form a (nonempty) complete lattice.
\end{enumerate}

\subsection{Inne}
\begin{enumerate}
 \item  A differentiable function $f$ is (strictly) concave on an interval if and only if its derivative function $f'$ is (strictly) monotonically decreasing on that interval, that is, a concave function has a non-increasing (decreasing) slope.
 \item If $f$ is twice-differentiable, then $f$ is concave if and only if $f''$ is non-positive (or, if the acceleration is non-positive). If its second derivative is negative then it is strictly concave, but the opposite is not true.
\end{enumerate}

\end{document}
/* vim: set ts=2 sw=2 tw=2 et : */
